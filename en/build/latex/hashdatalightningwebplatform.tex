%% Generated by Sphinx.
\def\sphinxdocclass{report}
\documentclass[letterpaper,10pt,english]{sphinxmanual}
\ifdefined\pdfpxdimen
   \let\sphinxpxdimen\pdfpxdimen\else\newdimen\sphinxpxdimen
\fi \sphinxpxdimen=.75bp\relax
\ifdefined\pdfimageresolution
    \pdfimageresolution= \numexpr \dimexpr1in\relax/\sphinxpxdimen\relax
\fi
%% let collapsible pdf bookmarks panel have high depth per default
\PassOptionsToPackage{bookmarksdepth=5}{hyperref}
%% turn off hyperref patch of \index as sphinx.xdy xindy module takes care of
%% suitable \hyperpage mark-up, working around hyperref-xindy incompatibility
\PassOptionsToPackage{hyperindex=false}{hyperref}
%% memoir class requires extra handling
\makeatletter\@ifclassloaded{memoir}
{\ifdefined\memhyperindexfalse\memhyperindexfalse\fi}{}\makeatother

\PassOptionsToPackage{booktabs}{sphinx}
\PassOptionsToPackage{colorrows}{sphinx}

\PassOptionsToPackage{warn}{textcomp}

\catcode`^^^^00a0\active\protected\def^^^^00a0{\leavevmode\nobreak\ }
\usepackage{cmap}
\usepackage{fontspec}
\defaultfontfeatures[\rmfamily,\sffamily,\ttfamily]{}
\usepackage{amsmath,amssymb,amstext}
\usepackage{polyglossia}
\setmainlanguage{english}



\setmainfont{FreeSerif}[
  Extension      = .otf,
  UprightFont    = *,
  ItalicFont     = *Italic,
  BoldFont       = *Bold,
  BoldItalicFont = *BoldItalic
]
\setsansfont{FreeSans}[
  Extension      = .otf,
  UprightFont    = *,
  ItalicFont     = *Oblique,
  BoldFont       = *Bold,
  BoldItalicFont = *BoldOblique,
]
\setmonofont{FreeMono}[
  Extension      = .otf,
  UprightFont    = *,
  ItalicFont     = *Oblique,
  BoldFont       = *Bold,
  BoldItalicFont = *BoldOblique,
]



\usepackage[Bjarne]{fncychap}
\usepackage{sphinx}

\fvset{fontsize=\small}
\usepackage{geometry}


% Include hyperref last.
\usepackage{hyperref}
% Fix anchor placement for figures with captions.
\usepackage{hypcap}% it must be loaded after hyperref.
% Set up styles of URL: it should be placed after hyperref.
\urlstyle{same}

\addto\captionsenglish{\renewcommand{\contentsname}{Contents}}

\usepackage{sphinxmessages}



\usepackage{longtable}
\usepackage{xeCJK}
\usepackage{graphicx}
\usepackage{amsmath, amssymb}
\usepackage{booktabs}
\usepackage{multirow}
\usepackage{titlesec}
\usepackage{tocloft}
% \newcommand{\sectionbreak}{\clearpage}
\renewcommand{\cftsecnumwidth}{2.5em}
\renewcommand{\cftsubsecnumwidth}{3.5em}
\renewcommand{\cftsubsubsecnumwidth}{4.5em}

% \titleformat{\chapter}[block]{\LARGE\bfseries}{\thechapter}{1em}{}
% \titleformat{\section}[block]{\Large\bfseries}{\thesection}{1em}{}
% \titleformat{\subsection}[block]{\large\bfseries}{\thesubsection}{1em}{}
% \titleformat{\subsubsection}[block]{\normalsize\bfseries}{\thesubsubsection}{1em}{}
\setcounter{secnumdepth}{3}
\setcounter{tocdepth}{3}
\usepackage[document]{ragged2e}
\usepackage{titlesec}

% 设置一级标题(section)带有序号
\titleformat{\section}{\normalfont\Large\bfseries}{\thesection}{1em}{}

% 设置二级标题(subsection)带有序号
\titleformat{\subsection}{\normalfont\large\bfseries}{\thesubsection}{1em}{}

% 设置三级标题(subsubsection)不带序号
\titleformat{\subsubsection}[block]{\normalfont\normalsize\bfseries}{}{0pt}{}

% 取消三级以下标题的序号
\setcounter{secnumdepth}{1}

% Custom LaTeX preamble
\usepackage{etoolbox}
\makeatletter
% Override \sphinxmaketitle to remove section numbering
\patchcmd{\sphinxmaketitle}
  {\section{\MakeUppercase{\@title}}}
  {\chapter*{\MakeUppercase{\@title}}}
  {}
  {}
\makeatother

\usepackage{multirow}


\title{HashData Lightning Web Platform Documentation}
\date{Jul 12, 2024}
\release{v1.5.4}
\author{HashData}
\newcommand{\sphinxlogo}{\vbox{}}
\renewcommand{\releasename}{Release}
\makeindex
\begin{document}

\pagestyle{empty}
\sphinxmaketitle
\pagestyle{plain}
\sphinxtableofcontents
\pagestyle{normal}
\phantomsection\label{\detokenize{index::doc}}


\sphinxstepscope


\chapter{Deploy HashData Lightning through Visual Interface}
\label{\detokenize{deploy-guides/physical-deploy/visualized-deploy:deploy-hashdata-lightning-through-visual-interface}}\label{\detokenize{deploy-guides/physical-deploy/visualized-deploy::doc}}
\sphinxAtStartPar
HashData Lightning Web Platform is a console tool for deploying and managing HashData Lightning clusters, providing a simple and intuitive user interface. Compared to the manual method, deployment through visual interface is simpler and more intuitive. You only need to follow the interface prompts to operate, without understanding complex commands and configuration files, making deployment more efficient.


\section{Applicable version}
\label{\detokenize{deploy-guides/physical-deploy/visualized-deploy:applicable-version}}
\sphinxAtStartPar
Make sure that you are deploying HashData Lightning v1.5.4 or a later version.


\section{Deploy HashData Lightning cluster}
\label{\detokenize{deploy-guides/physical-deploy/visualized-deploy:deploy-hashdata-lightning-cluster}}
\sphinxAtStartPar
This section introduces how to deploy HashData Lightning on physical machines using the Web Platform.


\subsection{Software and hardware configuration requirements}
\label{\detokenize{deploy-guides/physical-deploy/visualized-deploy:software-and-hardware-configuration-requirements}}
\sphinxAtStartPar
HashData Lightning supports deployment on the following operating systems and CPU architectures. See the table below for details.


\begin{savenotes}\sphinxattablestart
\sphinxthistablewithglobalstyle
\raggedright
\begin{tabulary}{\linewidth}[t]{TT}
\sphinxtoprule
\sphinxstyletheadfamily 
\sphinxAtStartPar
Operating system
&\sphinxstyletheadfamily 
\sphinxAtStartPar
Supported CPU architectures
\\
\sphinxmidrule
\sphinxtableatstartofbodyhook
\sphinxAtStartPar
RHEL/CentOS 7.6+
&
\sphinxAtStartPar
x86\_64 and AArch64
\\
\sphinxhline
\sphinxAtStartPar
Kylin V10 SP1 or SP2
&
\sphinxAtStartPar
x86\_64 and AArch64
\\
\sphinxbottomrule
\end{tabulary}
\sphinxtableafterendhook\par
\sphinxattableend\end{savenotes}


\subsection{Installation steps}
\label{\detokenize{deploy-guides/physical-deploy/visualized-deploy:installation-steps}}
\sphinxAtStartPar
Installing HashData Lightning on servers mainly involves 4 steps: preparation, installation of the database RPM package, database deployment, and post\sphinxhyphen{}installation setup.


\subsubsection{Step 1: Prepare to deploy}
\label{\detokenize{deploy-guides/physical-deploy/visualized-deploy:step-1-prepare-to-deploy}}\begin{enumerate}
\sphinxsetlistlabels{\arabic}{enumi}{enumii}{}{.}%
\item {} 
\sphinxAtStartPar
Before installation, check and confirm the basic information of the server to better plan and deploy the cluster.


\begin{savenotes}\sphinxattablestart
\sphinxthistablewithglobalstyle
\raggedright
\begin{tabular}[t]{\X{8}{36}\X{10}{36}\X{18}{36}}
\sphinxtoprule
\sphinxstyletheadfamily 
\sphinxAtStartPar
\sphinxstylestrong{Step}
&\sphinxstyletheadfamily 
\sphinxAtStartPar
\sphinxstylestrong{Command}
&\sphinxstyletheadfamily 
\sphinxAtStartPar
\sphinxstylestrong{Purpose}
\\
\sphinxmidrule
\sphinxtableatstartofbodyhook
\sphinxAtStartPar
1
&
\sphinxAtStartPar
\sphinxcode{\sphinxupquote{free \sphinxhyphen{}h}}
&
\sphinxAtStartPar
Views memory information of the operating system.
\\
\sphinxhline
\sphinxAtStartPar
2
&
\sphinxAtStartPar
\sphinxcode{\sphinxupquote{df \sphinxhyphen{}h}}
&
\sphinxAtStartPar
Checks disk space.
\\
\sphinxhline
\sphinxAtStartPar
3
&
\sphinxAtStartPar
\sphinxcode{\sphinxupquote{lscpu}}
&
\sphinxAtStartPar
Checks the number of CPUs.
\\
\sphinxhline
\sphinxAtStartPar
4
&
\sphinxAtStartPar
\sphinxcode{\sphinxupquote{cat/etc/system\sphinxhyphen{}release}}
&
\sphinxAtStartPar
Views the version information of the operating system.
\\
\sphinxhline
\sphinxAtStartPar
5
&
\sphinxAtStartPar
\sphinxcode{\sphinxupquote{uname \sphinxhyphen{}a}}
&
\sphinxAtStartPar
Outputs all kernel information in this order (where if the detection results of \sphinxhyphen{}p and \sphinxhyphen{}i are unknown, they are omitted): kernel name, hostname on the network node, kernel release number, kernel version, hardware architecture name of the host, processor type, hardware platform, operating system name.
\\
\sphinxhline
\sphinxAtStartPar
6
&
\sphinxAtStartPar
\sphinxcode{\sphinxupquote{tail \sphinxhyphen{}11 /proc/cpuinfo}}
&
\sphinxAtStartPar
Views CPU information.
\\
\sphinxbottomrule
\end{tabular}
\sphinxtableafterendhook\par
\sphinxattableend\end{savenotes}

\item {} 
\sphinxAtStartPar
Create a \sphinxcode{\sphinxupquote{gpadmin}} user on each node server as the admin user. Create a user group and username \sphinxcode{\sphinxupquote{gpadmin}}, and set the identification number of the user group and username to \sphinxcode{\sphinxupquote{520}}. Create and specify the home directory \sphinxcode{\sphinxupquote{/home/gpadmin/}}. The commands are as follows.

\begin{sphinxVerbatim}[commandchars=\\\{\}]
groupadd\PYG{+w}{ }\PYGZhy{}g\PYG{+w}{ }\PYG{l+m}{520}\PYG{+w}{ }gpadmin\PYG{+w}{  }\PYG{c+c1}{\PYGZsh{} Creates the user group gpadmin.}
useradd\PYG{+w}{ }\PYGZhy{}g\PYG{+w}{ }\PYG{l+m}{520}\PYG{+w}{ }\PYGZhy{}u\PYG{+w}{ }\PYG{l+m}{520}\PYG{+w}{ }\PYGZhy{}m\PYG{+w}{ }\PYGZhy{}d\PYG{+w}{ }/home/gpadmin/\PYG{+w}{ }\PYGZhy{}s\PYG{+w}{ }/bin/bash\PYG{+w}{ }gpadmin\PYG{+w}{  }\PYG{c+c1}{\PYGZsh{} Creates the user gpadmin and the home directory.}
passwd\PYG{+w}{ }gpadmin\PYG{+w}{  }\PYG{c+c1}{\PYGZsh{} Sets password for gpadmin.}
\end{sphinxVerbatim}

\item {} 
\sphinxAtStartPar
Configure as follows on the machine where HashData Lightning is to be installed:
\begin{enumerate}
\sphinxsetlistlabels{\arabic}{enumii}{enumiii}{}{.}%
\item {} 
\sphinxAtStartPar
Turn off the firewall. Otherwise, you cannot deploy through the Web Platform.

\begin{sphinxVerbatim}[commandchars=\\\{\}]
sudo\PYG{+w}{ }systemctl\PYG{+w}{ }stop\PYG{+w}{ }firewalld.service
sudo\PYG{+w}{ }systemctl\PYG{+w}{ }disable\PYG{+w}{ }firewalld.service
\end{sphinxVerbatim}

\item {} 
\sphinxAtStartPar
Disable SELinux. You can edit the \sphinxcode{\sphinxupquote{/etc/selinux/config}} file and set the value of SELINUX to \sphinxcode{\sphinxupquote{disabled}}:

\begin{sphinxVerbatim}[commandchars=\\\{\}]
sudo\PYG{+w}{ }sed\PYG{+w}{ }s/\PYGZca{}SELINUX\PYG{o}{=}.*\PYGZdl{}/SELINUX\PYG{o}{=}disabled/\PYG{+w}{ }\PYGZhy{}i\PYG{+w}{ }/etc/selinux/config
sudo\PYG{+w}{ }setenforce\PYG{+w}{ }\PYG{l+m}{0}
\end{sphinxVerbatim}

\end{enumerate}

\item {} 
\sphinxAtStartPar
Set system parameters on each server.

\item {} 
\sphinxAtStartPar
Permanently disable IPv6.

\sphinxAtStartPar
To do that, you need to edit the \sphinxcode{\sphinxupquote{/etc/sysctl.conf}} file (or create a new configuration file in the \sphinxcode{\sphinxupquote{/etc/sysctl.d/}} directory) and add the following line:

\begin{sphinxVerbatim}[commandchars=\\\{\}]
\PYG{n}{net}\PYG{o}{.}\PYG{n}{ipv6}\PYG{o}{.}\PYG{n}{conf}\PYG{o}{.}\PYG{n}{all}\PYG{o}{.}\PYG{n}{disable\PYGZus{}ipv6} \PYG{o}{=} \PYG{l+m+mi}{1}
\PYG{n}{net}\PYG{o}{.}\PYG{n}{ipv6}\PYG{o}{.}\PYG{n}{conf}\PYG{o}{.}\PYG{n}{default}\PYG{o}{.}\PYG{n}{disable\PYGZus{}ipv6} \PYG{o}{=} \PYG{l+m+mi}{1}
\end{sphinxVerbatim}

\sphinxAtStartPar
After that, run \sphinxcode{\sphinxupquote{sudo sysctl \sphinxhyphen{}p}} to apply the changes, or restart your system.

\item {} 
\sphinxAtStartPar
Configure password\sphinxhyphen{}free login between servers. Enable password\sphinxhyphen{}free login to each machine from other nodes in the \sphinxcode{\sphinxupquote{gpadmin}} account. The check command is \sphinxcode{\sphinxupquote{ssh ip}}, such as \sphinxcode{\sphinxupquote{ssh 192.168.48.58}}. If the password\sphinxhyphen{}free setting is successful, no password is required.

\item {} 
\sphinxAtStartPar
Enable the gpadmin user to perform \sphinxcode{\sphinxupquote{sudo}} without password.

\begin{sphinxadmonition}{note}{Note:}
\sphinxAtStartPar
After switching to the \sphinxcode{\sphinxupquote{gpadmin}} user by running \sphinxcode{\sphinxupquote{su \sphinxhyphen{} gpadmin}}, if you cannot run the \sphinxcode{\sphinxupquote{ifconfig}} command, you need to configure the environment variable for \sphinxcode{\sphinxupquote{ifconfig}}. Assuming the \sphinxcode{\sphinxupquote{ifconfig}} file is in the \sphinxcode{\sphinxupquote{/usr/sbin}} directory, you need to add a line \sphinxcode{\sphinxupquote{export PATH=/usr/sbin:\$PATH}} in the \sphinxcode{\sphinxupquote{\textasciitilde{}/.bashrc}} file, and then run \sphinxcode{\sphinxupquote{source \textasciitilde{}/.bashrc}} to make it effective.
\end{sphinxadmonition}

\item {} 
\sphinxAtStartPar
Copy the RPM package. Copy the RPM package to each node server where you want to install HashData Lightning.

\end{enumerate}


\subsubsection{Step 2: Install the database RPM package}
\label{\detokenize{deploy-guides/physical-deploy/visualized-deploy:step-2-install-the-database-rpm-package}}
\sphinxAtStartPar
On each node machine, run the following commands to install the database RPM package, and the system dependencies will be automatically installed.

\begin{sphinxVerbatim}[commandchars=\\\{\}]
\PYG{n+nb}{cd}\PYG{+w}{ }/home/gpadmin
sudo\PYG{+w}{ }yum\PYG{+w}{ }install\PYG{+w}{ }hashdata\PYGZhy{}lightning\PYGZhy{}1.5.4\PYGZhy{}1.el7.x86\PYGZus{}64\PYGZhy{}75889\PYGZhy{}release.rpm
sudo\PYG{+w}{ }chown\PYG{+w}{ }\PYGZhy{}R\PYG{+w}{ }gpadmin:gpadmin\PYG{+w}{ }/usr/local
sudo\PYG{+w}{ }chown\PYG{+w}{ }\PYGZhy{}R\PYG{+w}{ }gpadmin:gpadmin\PYG{+w}{ }/usr/local/cloudberry*
\end{sphinxVerbatim}

\begin{sphinxadmonition}{note}{Note:}
\sphinxAtStartPar
During the actual installation process, you need to replace the RPM file name \sphinxcode{\sphinxupquote{hashdata\sphinxhyphen{}lightning\sphinxhyphen{}1.5.4\sphinxhyphen{}1.el7.x86\_64\sphinxhyphen{}75889\sphinxhyphen{}release.rpm}} with the real RPM package name.
\end{sphinxadmonition}


\subsubsection{Step 3: Deploy the database through the interface}
\label{\detokenize{deploy-guides/physical-deploy/visualized-deploy:step-3-deploy-the-database-through-the-interface}}
\sphinxAtStartPar
Use Web Platform, the embedded visual interface, to deploy HashData Lightning. By default, the visual deployment tool accesses the \sphinxcode{\sphinxupquote{7788}} port of the database node server. After installation, the \sphinxcode{\sphinxupquote{7788}} port is open by default for all nodes.


\paragraph{Access the deployment interface}
\label{\detokenize{deploy-guides/physical-deploy/visualized-deploy:access-the-deployment-interface}}\begin{enumerate}
\sphinxsetlistlabels{\arabic}{enumi}{enumii}{}{.}%
\item {} 
\sphinxAtStartPar
Visit the deployment visual interface. Open your browser (IE series browsers are not supported) and visit the following link to open the interface. You need to replace \sphinxcode{\sphinxupquote{<IP>}} with the IP address of any node server:

\begin{sphinxVerbatim}[commandchars=\\\{\}]
\PYG{n}{http}\PYG{p}{:}\PYG{o}{/}\PYG{o}{/}\PYG{o}{\PYGZlt{}}\PYG{n}{IP}\PYG{o}{\PYGZgt{}}\PYG{p}{:}\PYG{l+m+mi}{7788}\PYG{o}{/}
\end{sphinxVerbatim}

\item {} 
\sphinxAtStartPar
Fill in the superuser password to log in to the deployment node, as shown in the following figure. To view the superuser password, run the command \sphinxcode{\sphinxupquote{find / \sphinxhyphen{}path "*/cloudberry\sphinxhyphen{}*/cloudberryUI/resources/users.json" 2>/dev/null | xargs cat | grep \sphinxhyphen{}A1 '"username": "gpmon",'}}.

\sphinxAtStartPar
The default installation directory is /usr/local, and you can view the username and password of the gpmon account using the command \sphinxcode{\sphinxupquote{cat /usr/local/cloudberry\sphinxhyphen{}db/cloudberryUI/resources/users.json}}.

\noindent\sphinxincludegraphics{{web-platform-deploy-login}.png}

\end{enumerate}

\sphinxAtStartPar
After successful login, choose the deployment mode: single\sphinxhyphen{}node deployment or multi\sphinxhyphen{}node deployment.

\begin{sphinxadmonition}{note}{Note:}
\sphinxAtStartPar
You cannot log in with the same IP address and user at the same time. Otherwise, an error will be prompted.
\end{sphinxadmonition}


\paragraph{Deploy in single\sphinxhyphen{}node mode}
\label{\detokenize{deploy-guides/physical-deploy/visualized-deploy:deploy-in-single-node-mode}}
\sphinxAtStartPar
The single\sphinxhyphen{}node deployment mode is intended for testing purposes. This mode does not support high availability. Do not use this mode in production environments.

\sphinxAtStartPar
This mode only requires one physical machine because all services will be deployed on the same machine.
\begin{enumerate}
\sphinxsetlistlabels{\arabic}{enumi}{enumii}{}{.}%
\item {} 
\sphinxAtStartPar
Once logged in, select \sphinxstylestrong{Single Node Deployment} and click \sphinxstylestrong{Next}.

\item {} 
\sphinxAtStartPar
Set the configuration items for a single node. The screenshot below shows an example configuration:

\noindent\sphinxincludegraphics{{web-platform-deploy-single-node}.png}

\item {} 
\sphinxAtStartPar
Click \sphinxstylestrong{Perform Deployment} and wait for the deployment to complete.

\sphinxAtStartPar
After the deployment is complete, you will see the following screen:

\noindent\sphinxincludegraphics{{web-platform-welcome}.png}

\end{enumerate}


\paragraph{Deploy in multi\sphinxhyphen{}node mode}
\label{\detokenize{deploy-guides/physical-deploy/visualized-deploy:deploy-in-multi-node-mode}}\begin{enumerate}
\sphinxsetlistlabels{\arabic}{enumi}{enumii}{}{.}%
\item {} 
\sphinxAtStartPar
Once logged in, select \sphinxstylestrong{Add nodes and start database cluster} and click Next.

\item {} 
\sphinxAtStartPar
Add a node. You can use the \sphinxstylestrong{Add in batch} to add nodes quickly, or you can add a node manually.
\begin{itemize}
\item {} 
\sphinxAtStartPar
To quickly add nodes: The deployment tool will automatically detect all nodes that have the RPM packages installed and show the \sphinxstylestrong{one\sphinxhyphen{}click add} in the upper\sphinxhyphen{}left corner of the window.

\sphinxAtStartPar
Click \sphinxstylestrong{one\sphinxhyphen{}click add} and the deployment tool will automatically add the available nodes.

\item {} 
\sphinxAtStartPar
To manually add nodes: Enter the hostname or IP address of the node that you want to add in the text box, such as \sphinxcode{\sphinxupquote{i\sphinxhyphen{}uv2qw6ad}} or \sphinxcode{\sphinxupquote{192.168.176.29}}, and then click \sphinxstylestrong{Add node}.

\begin{sphinxadmonition}{note}{Note:}\begin{itemize}
\item {} 
\sphinxAtStartPar
Make sure that the nodes you add can be detected and are not duplicated. Otherwise, the deployment tool will report an error at the top of the window, indicating that the hostname was not detected or the node to be added already exists.

\item {} 
\sphinxAtStartPar
The multi\sphinxhyphen{}node deployment mode cannot proceed if you only add one node.

\end{itemize}
\end{sphinxadmonition}

\end{itemize}

\item {} 
\sphinxAtStartPar
Complete the following configuration for the cluster:
\begin{itemize}
\item {} 
\sphinxAtStartPar
Configure the standby node for the primary node and configure mirror nodes for the data nodes.

\item {} 
\sphinxAtStartPar
\sphinxstylestrong{Data mirroring} determines whether the cluster’s data nodes have mirror copies. It is recommended to enable this option in production environments to ensure high availability.

\item {} 
\sphinxAtStartPar
Change the \sphinxcode{\sphinxupquote{gpmon}} password and check \sphinxstylestrong{Allow remote connection to the database}.

\end{itemize}

\noindent\sphinxincludegraphics{{web-platform-deploy-multi}.png}

\item {} 
\sphinxAtStartPar
Set the storage path. Note that the current HashData Lightning version requires the mounting points of all nodes to be specified to the same one. Otherwise, an error message is prompted. Then click \sphinxstylestrong{Next}.

\item {} 
\sphinxAtStartPar
Confirm the configurations made in the previous steps. You can go back to correct the wrong setting if there is one. Click \sphinxstylestrong{Start Deployment} in the lower\sphinxhyphen{}right corner. The deployment starts and a progress bar is displayed.

\sphinxAtStartPar
If the deployment is completed, you will be taken to the completion page. Note that you will be asked if you want to deploy again the next time you log in.

\item {} 
\sphinxAtStartPar
Run \sphinxcode{\sphinxupquote{psql}} to check whether the database is up. If yes, you can continue with the post\sphinxhyphen{}installation configuration. If not, try to log into the node server again and run \sphinxcode{\sphinxupquote{psql}} as the \sphinxcode{\sphinxupquote{gpadmin}} user.

\end{enumerate}


\subsubsection{Step 4: Perform post\sphinxhyphen{}installation configuration}
\label{\detokenize{deploy-guides/physical-deploy/visualized-deploy:step-4-perform-post-installation-configuration}}\begin{itemize}
\item {} 
\sphinxAtStartPar
Run the following command as the \sphinxcode{\sphinxupquote{gpadmin}} user.

\begin{sphinxVerbatim}[commandchars=\\\{\}]
sudo\PYG{+w}{ }chown\PYG{+w}{ }\PYGZhy{}R\PYG{+w}{ }gpadmin:gpadmin\PYG{+w}{ }/usr/local/cloudberry\PYGZhy{}db/cloudberryUI/resources
\end{sphinxVerbatim}

\item {} 
\sphinxAtStartPar
Enable remote connection.

\sphinxAtStartPar
HashData Lightning supports remote connections. If \sphinxstylestrong{Allow remote connection to database} is not checked when you configure the cluster parameters (as described in Step 3 of the above \sphinxstylestrong{Deploy in multi\sphinxhyphen{}node mode} section), you can add the following code line to the \sphinxcode{\sphinxupquote{\$COORDINATOR\_DATA\_DIRECTORY/pg\_hba}} file to allow users from any IP to connect through password authentication.

\sphinxAtStartPar
To ensure security, restrict the IP range or database name based on actual needs. For \sphinxcode{\sphinxupquote{pg\_hba.conf}}, the HashData technical support team has an auto\sphinxhyphen{}generated initialization version. The support engineers will configure the version on\sphinxhyphen{}site based on the actual situation and security requirements. It is recommended to check \sphinxcode{\sphinxupquote{pg\_hba.conf}}.

\begin{sphinxVerbatim}[commandchars=\\\{\}]
host\PYG{+w}{  }all\PYG{+w}{       }all\PYG{+w}{   }\PYG{l+m}{0}.0.0.0/0\PYG{+w}{  }md5
\end{sphinxVerbatim}

\sphinxAtStartPar
Once the changes are made, run the following command to reload the database configuration file \sphinxcode{\sphinxupquote{pg\_hba.conf}}:

\begin{sphinxVerbatim}[commandchars=\\\{\}]
gpstop\PYG{+w}{ }\PYGZhy{}u
\end{sphinxVerbatim}

\item {} 
\sphinxAtStartPar
You can use the following commands to start, stop, restart, and view the status of HashData Lightning.


\begin{savenotes}\sphinxattablestart
\sphinxthistablewithglobalstyle
\raggedright
\begin{tabular}[t]{\X{8}{26}\X{18}{26}}
\sphinxtoprule
\sphinxstyletheadfamily 
\sphinxAtStartPar
Command
&\sphinxstyletheadfamily 
\sphinxAtStartPar
Description
\\
\sphinxmidrule
\sphinxtableatstartofbodyhook
\sphinxAtStartPar
\sphinxcode{\sphinxupquote{gpstop \sphinxhyphen{}a}}
&
\sphinxAtStartPar
Stops the cluster. In this mode, if there is a connected session, you need to wait for the session to be closed before stopping the cluster.
\\
\sphinxhline
\sphinxAtStartPar
\sphinxcode{\sphinxupquote{gpstop \sphinxhyphen{}af}}
&
\sphinxAtStartPar
Forcibly shuts down the cluster.
\\
\sphinxhline
\sphinxAtStartPar
\sphinxcode{\sphinxupquote{gpstop \sphinxhyphen{}ar}}
&
\sphinxAtStartPar
Restarts the cluster. Waits for the SQL statement to finish execution. In this mode, if there is a connected session, you need to wait for the session to be closed before stopping the cluster.
\\
\sphinxhline
\sphinxAtStartPar
\sphinxcode{\sphinxupquote{gpstate \sphinxhyphen{}s}}
&
\sphinxAtStartPar
Shows the current status of the cluster.
\\
\sphinxbottomrule
\end{tabular}
\sphinxtableafterendhook\par
\sphinxattableend\end{savenotes}

\end{itemize}


\section{Troubleshooting tips}
\label{\detokenize{deploy-guides/physical-deploy/visualized-deploy:troubleshooting-tips}}\begin{itemize}
\item {} 
\sphinxAtStartPar
After logging into the console through \sphinxcode{\sphinxupquote{http://<IP>:7788/}}, if you see a message indicating that the cluster nodes are not connected or stuck in the process of collecting host information, it is recommended to check that the SSH mutual trust between the nodes is properly configured, and then run the following commands to restart the node:
\begin{quote}

\begin{sphinxVerbatim}[commandchars=\\\{\}]
su\PYG{+w}{ }\PYGZhy{}\PYG{+w}{ }gpadmin
\PYG{n+nb}{cd}\PYG{+w}{ }/usr/local/cloudberry\PYGZhy{}db
sudo\PYG{+w}{ }pkill\PYG{+w}{ }cbuiserver
./cbuiserver
\end{sphinxVerbatim}
\end{quote}

\item {} 
\sphinxAtStartPar
If the node machines have previously undergone visual deployment and you wish to reinstall the RPM packages on these machines, run \sphinxcode{\sphinxupquote{sudo pkill cbuiserver}} on each machine before installation and then clear the \sphinxcode{\sphinxupquote{/usr/local/cloudberry\sphinxhyphen{}db}} directory.

\end{itemize}

\sphinxstepscope


\chapter{View monitoring data using the Web Platform}
\label{\detokenize{manage-system/web-platform-monitoring/web-platform-monitoring-index:view-monitoring-data-using-the-web-platform}}\label{\detokenize{manage-system/web-platform-monitoring/web-platform-monitoring-index::doc}}
\sphinxstepscope


\section{View Cluster Information}
\label{\detokenize{manage-system/web-platform-monitoring/web-platform-view-cluster-overview:view-cluster-information}}\label{\detokenize{manage-system/web-platform-monitoring/web-platform-view-cluster-overview::doc}}

\subsection{Steps}
\label{\detokenize{manage-system/web-platform-monitoring/web-platform-view-cluster-overview:steps}}\begin{enumerate}
\sphinxsetlistlabels{\arabic}{enumi}{enumii}{}{.}%
\item {} 
\sphinxAtStartPar
Access \sphinxcode{\sphinxupquote{http://<ip>:7788/}} to log into the Web Platform console.

\item {} 
\sphinxAtStartPar
Click \sphinxstylestrong{Dashboard} in the left navigation menu to view the cluster overview.

\noindent\sphinxincludegraphics{{web-platform-dashboard}.png}


\begin{savenotes}\sphinxattablestart
\sphinxthistablewithglobalstyle
\raggedright
\begin{tabular}[t]{\X{10}{29}\X{19}{29}}
\sphinxtoprule
\sphinxstyletheadfamily 
\sphinxAtStartPar
Display item
&\sphinxstyletheadfamily 
\sphinxAtStartPar
Description
\\
\sphinxmidrule
\sphinxtableatstartofbodyhook
\sphinxAtStartPar
Overview
&
\sphinxAtStartPar
Cluster version, console version, cluster uptime, the number of cluster connection sessions, the total number of databases in the cluster, and total number of tables.
\\
\sphinxhline
\sphinxAtStartPar
Database State
&
\sphinxAtStartPar
The overall status of the database and the number of normal and abnormal segments.
\\
\sphinxhline
\sphinxAtStartPar
Disk Usage Summary
&
\sphinxAtStartPar
Disk usage of the coordinator and segments.
\\
\sphinxbottomrule
\end{tabular}
\sphinxtableafterendhook\par
\sphinxattableend\end{savenotes}

\noindent\sphinxincludegraphics{{web-platform-dashboard-metrics}.png}


\begin{savenotes}\sphinxattablestart
\sphinxthistablewithglobalstyle
\raggedright
\begin{tabular}[t]{\X{8}{28}\X{20}{28}}
\sphinxtoprule
\sphinxstyletheadfamily 
\sphinxAtStartPar
Display item
&\sphinxstyletheadfamily 
\sphinxAtStartPar
Description
\\
\sphinxmidrule
\sphinxtableatstartofbodyhook
\sphinxAtStartPar
CPU
&
\sphinxAtStartPar
Shows the average CPU and maximum CPU usage of the cluster in the past 2 hours.
\sphinxhyphen{} The orange and blue lines represent the system processes and user processes, respectively. Click the legend in the upper left corner to show or hide the line.
\sphinxhyphen{} Hover the cursor over the chart to display the CPU usage percentage at specific time points.
\\
\sphinxhline
\sphinxAtStartPar
Memory
&
\sphinxAtStartPar
Shows the average memory usage percentage of the cluster in the past 2 hours.
\sphinxhyphen{} Hover the cursor over the chart to display the average memory and maximum memory usage percentage at specific time points.
\\
\sphinxbottomrule
\end{tabular}
\sphinxtableafterendhook\par
\sphinxattableend\end{savenotes}

\end{enumerate}

\sphinxstepscope


\section{Cluster Status and Metrics}
\label{\detokenize{manage-system/web-platform-monitoring/web-platform-view-cluster-status:cluster-status-and-metrics}}\label{\detokenize{manage-system/web-platform-monitoring/web-platform-view-cluster-status::doc}}
\sphinxAtStartPar
You can view the near\sphinxhyphen{}real\sphinxhyphen{}time status and metric data of the cluster on the \sphinxstylestrong{Cluster Metrics} and \sphinxstylestrong{Cluster Status} pages.


\subsection{Access the pages}
\label{\detokenize{manage-system/web-platform-monitoring/web-platform-view-cluster-status:access-the-pages}}
\sphinxAtStartPar
To access the \sphinxstylestrong{Cluster Metrics} and \sphinxstylestrong{Cluster Status} pages, you need to:
\begin{enumerate}
\sphinxsetlistlabels{\arabic}{enumi}{enumii}{}{.}%
\item {} 
\sphinxAtStartPar
Access the Web Platform dashboard in your browser via \sphinxcode{\sphinxupquote{http://<cluster\_node\_IP>:7788/}}.

\item {} 
\sphinxAtStartPar
Click \sphinxstylestrong{Cluster Metrics} or \sphinxstylestrong{Cluster Status} in the left navigation menu to enter the page.

\end{enumerate}


\subsection{View the overall status and data of the cluster}
\label{\detokenize{manage-system/web-platform-monitoring/web-platform-view-cluster-status:view-the-overall-status-and-data-of-the-cluster}}
\noindent\sphinxincludegraphics{{web-platform-view-cluster-status-1}.png}

\sphinxAtStartPar
The \sphinxstylestrong{Cluster Metrics} page displays the cluster’s CPU, memory, disk I/O, and network data.

\sphinxAtStartPar
You can adjust the time range of the data displayed through the drop\sphinxhyphen{}down menu on the upper\sphinxhyphen{}right corner of the page. Available time range options are “2 hours”, “6 hours”, “1 day” and “7 days”. Data calculation time units vary with the selected time range.

\sphinxAtStartPar
The charts have time as the X axis and numerical value or percentage as the Y axis. When you hover over a certain time point, the data of that time will pop up. You can view data from different time points by moving the cursor horizontally. Corresponding legends are provided in the upper right corner of each chart.


\subsubsection{Cluster CPU usage status}
\label{\detokenize{manage-system/web-platform-monitoring/web-platform-view-cluster-status:cluster-cpu-usage-status}}
\sphinxAtStartPar
The chart displays the following metrics:
\begin{itemize}
\item {} 
\sphinxAtStartPar
Average, skewness, and maximum value of the total CPU usage (\%) of all user processes.

\item {} 
\sphinxAtStartPar
Average, skewness, and maximum value of the total CPU usage (\%) of all system processes.

\item {} 
\sphinxAtStartPar
Average, skewness, and maximum of the total CPU usage (\%).

\item {} 
\sphinxAtStartPar
Name of the busiest host.

\end{itemize}


\subsubsection{Cluster memory usage status}
\label{\detokenize{manage-system/web-platform-monitoring/web-platform-view-cluster-status:cluster-memory-usage-status}}
\sphinxAtStartPar
The chart displays the following metrics:
\begin{itemize}
\item {} 
\sphinxAtStartPar
Average, skewness, and maximum of the total memory in use (\%).

\item {} 
\sphinxAtStartPar
Average, skewness, and maximum of the total buffer and cache memory (\%).

\item {} 
\sphinxAtStartPar
Average, skewness, and maximum value of the total available memory (\%).

\item {} 
\sphinxAtStartPar
Name of the busiest host

\end{itemize}


\subsubsection{Cluster disk I/O rate}
\label{\detokenize{manage-system/web-platform-monitoring/web-platform-view-cluster-status:cluster-disk-i-o-rate}}
\sphinxAtStartPar
The chart displays the following indicators:
\begin{itemize}
\item {} 
\sphinxAtStartPar
Average, skewness, and maximum value of the total disk read rate (MB/s).

\item {} 
\sphinxAtStartPar
Average, skewness, and maximum value of the total disk write rate (MB/s).

\item {} 
\sphinxAtStartPar
Name of the busiest host by disk read.

\item {} 
\sphinxAtStartPar
Name of the busiest host by disk write.

\end{itemize}

\begin{sphinxadmonition}{attention}{Attention:}
\sphinxAtStartPar
The part above the X axis of the chart displays disk read data, and the part below the X axis displays disk write data.
\end{sphinxadmonition}


\subsubsection{Network I/O rate}
\label{\detokenize{manage-system/web-platform-monitoring/web-platform-view-cluster-status:network-i-o-rate}}\begin{itemize}
\item {} 
\sphinxAtStartPar
Average, skewness, and maximum value of the total network read rate (MB/s).

\item {} 
\sphinxAtStartPar
Average, skewness, and maximum value of the total network write rate (MB/s).

\item {} 
\sphinxAtStartPar
Name of the busiest host by network read.

\item {} 
\sphinxAtStartPar
Name of the busiest host by network write.

\end{itemize}

\begin{sphinxadmonition}{note}{Note:}
\sphinxAtStartPar
The part above the X axis of the chart displays network read data, and the part below the X axis displays network write data.
\end{sphinxadmonition}


\subsection{View the status and data of nodes and hosts}
\label{\detokenize{manage-system/web-platform-monitoring/web-platform-view-cluster-status:view-the-status-and-data-of-nodes-and-hosts}}
\sphinxAtStartPar
The \sphinxstylestrong{Cluster Status} page displays the status and data of the hosts and nodes in the cluster. You can switch between host and segment tabs from the upper\sphinxhyphen{}left corner of the page.

\noindent\sphinxincludegraphics{{web-platform-view-cluster-status-2}.png}


\subsubsection{Host metrics}
\label{\detokenize{manage-system/web-platform-monitoring/web-platform-view-cluster-status:host-metrics}}
\sphinxAtStartPar
The \sphinxstylestrong{Host Metrics} tab displays data of the coordinator host, its backup host, and the segment hosts:

\begin{sphinxadmonition}{note}{Note:}
\sphinxAtStartPar
Users can search for a specific segment host by hostname.
\end{sphinxadmonition}


\begin{savenotes}\sphinxattablestart
\sphinxthistablewithglobalstyle
\raggedright
\begin{tabulary}{\linewidth}[t]{TT}
\sphinxtoprule
\sphinxstyletheadfamily 
\sphinxAtStartPar
Field
&\sphinxstyletheadfamily 
\sphinxAtStartPar
Description
\\
\sphinxmidrule
\sphinxtableatstartofbodyhook
\sphinxAtStartPar
Hostname
&
\sphinxAtStartPar
The name of the coordinator or segment host
\\
\sphinxhline
\sphinxAtStartPar
CPU User/System/Idle
&
\sphinxAtStartPar
\% of the user processes CPU usage, system processes CPU usage, and idle CPU
\\
\sphinxhline
\sphinxAtStartPar
Memory in Use (GB)
&
\sphinxAtStartPar
In\sphinxhyphen{}use and available memory
\\
\sphinxhline
\sphinxAtStartPar
Disk R (MB/s)
&
\sphinxAtStartPar
Disk read rate
\\
\sphinxhline
\sphinxAtStartPar
Disk W (MB/s)
&
\sphinxAtStartPar
Disk write rate
\\
\sphinxhline
\sphinxAtStartPar
Net R (MB/s)
&
\sphinxAtStartPar
Network read rate
\\
\sphinxhline
\sphinxAtStartPar
Net W (MB/s)
&
\sphinxAtStartPar
Network write rate
\\
\sphinxbottomrule
\end{tabulary}
\sphinxtableafterendhook\par
\sphinxattableend\end{savenotes}


\subsubsection{Segment status}
\label{\detokenize{manage-system/web-platform-monitoring/web-platform-view-cluster-status:segment-status}}
\sphinxAtStartPar
The \sphinxstylestrong{Segment Status} tab displays the status and data of each segment.

\sphinxAtStartPar
The top of the tab shows the overall status of the database, segment count, segment\sphinxhyphen{}specific host count, and a segment status indicator.

\sphinxAtStartPar
The table displays the following metrics and status:


\begin{savenotes}\sphinxattablestart
\sphinxthistablewithglobalstyle
\raggedright
\begin{tabulary}{\linewidth}[t]{TT}
\sphinxtoprule
\sphinxstyletheadfamily 
\sphinxAtStartPar
Field
&\sphinxstyletheadfamily 
\sphinxAtStartPar
Description
\\
\sphinxmidrule
\sphinxtableatstartofbodyhook
\sphinxAtStartPar
Hostname
&
\sphinxAtStartPar
The name of the segment host
\\
\sphinxhline
\sphinxAtStartPar
Address
&
\sphinxAtStartPar
The address of the segment on the segment host
\\
\sphinxhline
\sphinxAtStartPar
Port
&
\sphinxAtStartPar
The listening TCP port of the segment host
\\
\sphinxhline
\sphinxAtStartPar
DB ID
&
\sphinxAtStartPar
Database ID
\\
\sphinxhline
\sphinxAtStartPar
Content ID
&
\sphinxAtStartPar
The content identifier of the segment
\\
\sphinxhline
\sphinxAtStartPar
Status
&
\sphinxAtStartPar
Segment status. Values: Up or Down
\\
\sphinxhline
\sphinxAtStartPar
Role
&
\sphinxAtStartPar
The current role of the segment. Values: p (Primary) or m (Mirror)
\\
\sphinxhline
\sphinxAtStartPar
Preferred Role
&
\sphinxAtStartPar
The role that is set for the segment when it is initialized. Values: p (Primary) or m (Mirror)
\\
\sphinxhline
\sphinxAtStartPar
Replication Mode
&
\sphinxAtStartPar
The synchronization status of the segment with its mirror copy. Values: s (Synchronized) or n (Not In Sync)
\\
\sphinxbottomrule
\end{tabulary}
\sphinxtableafterendhook\par
\sphinxattableend\end{savenotes}

\sphinxstepscope


\section{View Database Object Information}
\label{\detokenize{manage-system/web-platform-monitoring/web-platform-view-db-object-info:view-database-object-information}}\label{\detokenize{manage-system/web-platform-monitoring/web-platform-view-db-object-info::doc}}
\sphinxAtStartPar
HashData Lightning Web Platform provides information of database objects. On the \sphinxstylestrong{Tables} page, you can view detailed information about tables in the database, such as the schema to which the table belongs, the table name, whether the table contains partitions, the table size, the user, and the estimated number of rows. An example page is as follows:

\noindent\sphinxincludegraphics{{web-platform-view-db-object-info-1}.png}

\sphinxAtStartPar
To access the \sphinxstylestrong{Tables} page, you need to:
\begin{enumerate}
\sphinxsetlistlabels{\arabic}{enumi}{enumii}{}{.}%
\item {} 
\sphinxAtStartPar
Access the Web Platform dashboard in your browser via \sphinxcode{\sphinxupquote{http://<cluster\_node\_IP>:7788/}}.

\item {} 
\sphinxAtStartPar
Click \sphinxstylestrong{Tables} in the left navigation menu to enter the page.

\end{enumerate}


\subsection{View table objects}
\label{\detokenize{manage-system/web-platform-monitoring/web-platform-view-db-object-info:view-table-objects}}
\sphinxAtStartPar
To query the information of a table, you can fill in the drop\sphinxhyphen{}down option boxes according to the database, the schema, the table name, and the user. Then click \sphinxstylestrong{Query}. The \sphinxstylestrong{User} box supports multiple selections.

\sphinxAtStartPar
The fields for the table information are described as follows:


\begin{savenotes}\sphinxattablestart
\sphinxthistablewithglobalstyle
\raggedright
\begin{tabulary}{\linewidth}[t]{TT}
\sphinxtoprule
\sphinxstyletheadfamily 
\sphinxAtStartPar
Field name
&\sphinxstyletheadfamily 
\sphinxAtStartPar
Description
\\
\sphinxmidrule
\sphinxtableatstartofbodyhook
\sphinxAtStartPar
Schema
&
\sphinxAtStartPar
The schema to which the table belongs.
\\
\sphinxhline
\sphinxAtStartPar
Relation Name
&
\sphinxAtStartPar
The table or view name.
\\
\sphinxhline
\sphinxAtStartPar
Include Partitions
&
\sphinxAtStartPar
Indicates whether the table contains partitions. \sphinxcode{\sphinxupquote{False}} means no, and \sphinxcode{\sphinxupquote{true}} means yes.
\\
\sphinxhline
\sphinxAtStartPar
Size
&
\sphinxAtStartPar
The storage size that the table occupies.
\\
\sphinxhline
\sphinxAtStartPar
User
&
\sphinxAtStartPar
The database user to which the table belongs.
\\
\sphinxhline
\sphinxAtStartPar
Est. Rows
&
\sphinxAtStartPar
The estimated number of rows.
\\
\sphinxhline
\sphinxAtStartPar
Last Analyzed
&
\sphinxAtStartPar
The last time the table was analyzed for updated statistics.
\\
\sphinxbottomrule
\end{tabulary}
\sphinxtableafterendhook\par
\sphinxattableend\end{savenotes}

\sphinxstepscope


\section{View SQL Monitoring Information}
\label{\detokenize{manage-system/web-platform-monitoring/web-platform-sql-monitor-info:view-sql-monitoring-information}}\label{\detokenize{manage-system/web-platform-monitoring/web-platform-sql-monitor-info::doc}}
\sphinxAtStartPar
HashData Lightning Web Platform provides monitoring information for SQL statements being executed in the database. On the \sphinxstylestrong{Query Monitor} page, you can see the execution status and details of each SQL statement,  and the status of each database session. An example is as follows:

\noindent\sphinxincludegraphics{{web-platform-view-sql-monitor-info-1}.png}


\subsection{Access the page}
\label{\detokenize{manage-system/web-platform-monitoring/web-platform-sql-monitor-info:access-the-page}}
\sphinxAtStartPar
To access the \sphinxstylestrong{Query Monitor} page, you need to:
\begin{enumerate}
\sphinxsetlistlabels{\arabic}{enumi}{enumii}{}{.}%
\item {} 
\sphinxAtStartPar
Access the Web Platform dashboard in your browser via \sphinxcode{\sphinxupquote{http://<cluster\_node\_IP>:7788/}}.

\item {} 
\sphinxAtStartPar
Click \sphinxstylestrong{Query Monitor} in the left navigation menu to enter the page.

\end{enumerate}


\subsection{View SQL execution status}
\label{\detokenize{manage-system/web-platform-monitoring/web-platform-sql-monitor-info:view-sql-execution-status}}
\sphinxAtStartPar
To view the execution status of a SQL statement, click the \sphinxstylestrong{Query Status} tab.

\sphinxAtStartPar
In the search area, fill in the corresponding drop\sphinxhyphen{}down option box according to the execution status of the SQL statement, the user, and the database. Then click \sphinxstylestrong{Query} to search. The \sphinxstylestrong{User} box supports multiple selections.

\sphinxAtStartPar
The options in the \sphinxstylestrong{Status} drop\sphinxhyphen{}down box are described as follows:


\begin{savenotes}\sphinxattablestart
\sphinxthistablewithglobalstyle
\raggedright
\begin{tabulary}{\linewidth}[t]{TT}
\sphinxtoprule
\sphinxstyletheadfamily 
\sphinxAtStartPar
Option name
&\sphinxstyletheadfamily 
\sphinxAtStartPar
Description
\\
\sphinxmidrule
\sphinxtableatstartofbodyhook
\sphinxAtStartPar
Running
&
\sphinxAtStartPar
The SQL statement is being executed.
\\
\sphinxhline
\sphinxAtStartPar
Done
&
\sphinxAtStartPar
The SQL statement has been executed.
\\
\sphinxhline
\sphinxAtStartPar
Abort
&
\sphinxAtStartPar
The SQL statement has been aborted.
\\
\sphinxhline
\sphinxAtStartPar
Canceling
&
\sphinxAtStartPar
SQL statement is being canceled.
\\
\sphinxbottomrule
\end{tabulary}
\sphinxtableafterendhook\par
\sphinxattableend\end{savenotes}

\sphinxAtStartPar
After clicking \sphinxstylestrong{Query}, a SQL list will be displayed in the area below, and you should be able to find the target SQL statement from the list. The fields of the SQL list are described as follows:


\begin{savenotes}\sphinxattablestart
\sphinxthistablewithglobalstyle
\raggedright
\begin{tabulary}{\linewidth}[t]{TT}
\sphinxtoprule
\sphinxstyletheadfamily 
\sphinxAtStartPar
Field name
&\sphinxstyletheadfamily 
\sphinxAtStartPar
Description
\\
\sphinxmidrule
\sphinxtableatstartofbodyhook
\sphinxAtStartPar
Query ID
&
\sphinxAtStartPar
Identifies the unique ID of the SQL statement being executed in the database.
\\
\sphinxhline
\sphinxAtStartPar
Status
&
\sphinxAtStartPar
The status of the SQL statement execution.
\\
\sphinxhline
\sphinxAtStartPar
User
&
\sphinxAtStartPar
The user who executes the SQL statement.
\\
\sphinxhline
\sphinxAtStartPar
Database
&
\sphinxAtStartPar
The database where the SQL statement is executed.
\\
\sphinxhline
\sphinxAtStartPar
Submitted Time
&
\sphinxAtStartPar
The time when the SQL statement was submitted.
\\
\sphinxhline
\sphinxAtStartPar
Queue Time
&
\sphinxAtStartPar
The queue time before executing the SQL statement.
\\
\sphinxhline
\sphinxAtStartPar
Run Time
&
\sphinxAtStartPar
The execution time of the SQL statement.
\\
\sphinxhline
\sphinxAtStartPar
Operation
&
\sphinxAtStartPar
For a SQL statement, you can click Cancel Query to cancel the SQL statement.
\\
\sphinxbottomrule
\end{tabulary}
\sphinxtableafterendhook\par
\sphinxattableend\end{savenotes}


\subsubsection{Cancel SQL execution}
\label{\detokenize{manage-system/web-platform-monitoring/web-platform-sql-monitor-info:cancel-sql-execution}}
\sphinxAtStartPar
To cancel one or more SQL statements, locate the \sphinxstylestrong{Operation} column of the corresponding SQL statement in the SQL list, and then click \sphinxstylestrong{Cancel Query}.

\noindent\sphinxincludegraphics{{web-platform-view-sql-monitor-info-2}.png}


\subsubsection{View SQL details}
\label{\detokenize{manage-system/web-platform-monitoring/web-platform-sql-monitor-info:view-sql-details}}
\sphinxAtStartPar
To view the details of a SQL statement, click the query ID of the SQL statement, and then enter the details page.

\noindent\sphinxincludegraphics{{web-platform-view-sql-monitor-info-3}.png}

\sphinxAtStartPar
The details page displays the details of SQL execution. You can click different tabs to view the query plan diagram, SQL text, and the query plan text of the SQL statement. An example is as follows:

\noindent\sphinxincludegraphics{{web-platform-view-sql-monitor-info-4}.png}


\subsection{View session status}
\label{\detokenize{manage-system/web-platform-monitoring/web-platform-sql-monitor-info:view-session-status}}
\sphinxAtStartPar
To view session status in the database, click the \sphinxstylestrong{Session Status} tab on the \sphinxstylestrong{Query Monitor} page.

\sphinxAtStartPar
A list of real\sphinxhyphen{}time sessions running in the database is displayed, including session ID, execution status, the user who operates, the database where the session is running, the start time, the application, and idle time.

\noindent\sphinxincludegraphics{{web-platform-view-sql-monitor-info-5}.png}

\sphinxAtStartPar
To view the details of a session, in the search area, fill in the corresponding drop\sphinxhyphen{}down option box according to the execution status, user, database, and application name. Then click \sphinxstylestrong{Query} to search. The \sphinxstylestrong{User} box supports multiple selections.

\sphinxAtStartPar
The options in the \sphinxstylestrong{Status} drop\sphinxhyphen{}down box are described as follows:


\begin{savenotes}\sphinxattablestart
\sphinxthistablewithglobalstyle
\raggedright
\begin{tabulary}{\linewidth}[t]{TT}
\sphinxtoprule
\sphinxstyletheadfamily 
\sphinxAtStartPar
Option name
&\sphinxstyletheadfamily 
\sphinxAtStartPar
Description
\\
\sphinxmidrule
\sphinxtableatstartofbodyhook
\sphinxAtStartPar
Active
&
\sphinxAtStartPar
The backend is running the session.
\\
\sphinxhline
\sphinxAtStartPar
Idle
&
\sphinxAtStartPar
The backend is waiting for new client commands.
\\
\sphinxhline
\sphinxAtStartPar
Idle in transaction (aborted)
&
\sphinxAtStartPar
The backend is in a transaction, but currently, no query is running.
\\
\sphinxhline
\sphinxAtStartPar
Fastpath function call
&
\sphinxAtStartPar
The backend is executing the fast path function.
\\
\sphinxhline
\sphinxAtStartPar
Disabled
&
\sphinxAtStartPar
The status is reported when \sphinxcode{\sphinxupquote{track\_activities}} is disabled in the backend.
\\
\sphinxhline
\sphinxAtStartPar
Unknown
&
\sphinxAtStartPar
The session status is unknown.
\\
\sphinxbottomrule
\end{tabulary}
\sphinxtableafterendhook\par
\sphinxattableend\end{savenotes}

\sphinxAtStartPar
After clicking \sphinxstylestrong{Query}, a list of sessions is displayed in the area below, and you should be able to find the target session from the list.

\sphinxAtStartPar
By default, the session list is sorted by \sphinxstylestrong{Start Time} in descending order. You can click \sphinxstylestrong{Start Time} to sort in ascending order, or sort by \sphinxstylestrong{Idle Time}. The description of the fields in the list is as follows:


\begin{savenotes}\sphinxattablestart
\sphinxthistablewithglobalstyle
\raggedright
\begin{tabulary}{\linewidth}[t]{TT}
\sphinxtoprule
\sphinxstyletheadfamily 
\sphinxAtStartPar
Option name
&\sphinxstyletheadfamily 
\sphinxAtStartPar
Description
\\
\sphinxmidrule
\sphinxtableatstartofbodyhook
\sphinxAtStartPar
Session ID
&
\sphinxAtStartPar
Identifies the unique ID of the session being executed in the database.
\\
\sphinxhline
\sphinxAtStartPar
Status
&
\sphinxAtStartPar
The status of the session.
\\
\sphinxhline
\sphinxAtStartPar
User
&
\sphinxAtStartPar
The user who performs the session operation.
\\
\sphinxhline
\sphinxAtStartPar
Database
&
\sphinxAtStartPar
The database where the session is running.
\\
\sphinxhline
\sphinxAtStartPar
Start Time
&
\sphinxAtStartPar
The start time of the session.
\\
\sphinxhline
\sphinxAtStartPar
Application
&
\sphinxAtStartPar
The client application for executing the session.
\\
\sphinxhline
\sphinxAtStartPar
Idle time
&
\sphinxAtStartPar
The idle time of the session.
\\
\sphinxhline
\sphinxAtStartPar
Operation
&
\sphinxAtStartPar
For running sessions, you can click \sphinxstylestrong{Cancel Query} to cancel the session.
\\
\sphinxbottomrule
\end{tabulary}
\sphinxtableafterendhook\par
\sphinxattableend\end{savenotes}

\sphinxstepscope


\section{View Storage Information}
\label{\detokenize{manage-system/web-platform-monitoring/web-platform-storage-overview:view-storage-information}}\label{\detokenize{manage-system/web-platform-monitoring/web-platform-storage-overview::doc}}

\subsection{Steps}
\label{\detokenize{manage-system/web-platform-monitoring/web-platform-storage-overview:steps}}\begin{enumerate}
\sphinxsetlistlabels{\arabic}{enumi}{enumii}{}{.}%
\item {} 
\sphinxAtStartPar
Access \sphinxcode{\sphinxupquote{http://<ip>:7788/}} to log into the Web Platform console.

\item {} 
\sphinxAtStartPar
Click \sphinxstylestrong{Storage Usage} in the left navigation menu to view the storage overview information.

\noindent\sphinxincludegraphics{{web-platform-view-storage-info-1}.png}

\item {} 
\sphinxAtStartPar
Click the \sphinxstylestrong{Coordinator} and \sphinxstylestrong{Segments} cards at the top of the page to view the disk usage of the machines hosting the coordinator and segment nodes respectively.


\begin{savenotes}\sphinxattablestart
\sphinxthistablewithglobalstyle
\raggedright
\begin{tabulary}{\linewidth}[t]{TT}
\sphinxtoprule
\sphinxstyletheadfamily 
\sphinxAtStartPar
Display item
&\sphinxstyletheadfamily 
\sphinxAtStartPar
Description
\\
\sphinxmidrule
\sphinxtableatstartofbodyhook
\sphinxAtStartPar
Host Name
&
\sphinxAtStartPar
Host names of the coordinator or segments. There may be multiple hosts.
\\
\sphinxhline
\sphinxAtStartPar
Data Directory
&
\sphinxAtStartPar
The mounting point and path information of each machine.
\\
\sphinxhline
\sphinxAtStartPar
Used Disk Space (\%)
&
\sphinxAtStartPar
The disk usage of each machine and its percentage.
\\
\sphinxhline
\sphinxAtStartPar
Disk Space Free (GB)
&
\sphinxAtStartPar
The available amount of disks on each machine and its percentage.
\\
\sphinxhline
\sphinxAtStartPar
Total Space (GB)
&
\sphinxAtStartPar
The total capacity of disks on each machine.
\\
\sphinxbottomrule
\end{tabulary}
\sphinxtableafterendhook\par
\sphinxattableend\end{savenotes}

\item {} 
\sphinxAtStartPar
Click the small triangle to the left of the \sphinxstylestrong{Hostname} to expand and view the disk usage under different mount points.

\end{enumerate}



\renewcommand{\indexname}{Index}
\printindex
\end{document}